\subsection{Budget control}
\label{ssec:budgetControl}
Even if we have taken into account all expenses, including the unforeseen ones, we still could have deviations on the budget due to different reasons: hardware failure, more costs of delays on development stages than expected in unforeseen costs, etc.

All hardware and software resources have a useful life time of 2 years or more, so even if we need more time to finish the project, we will not increase the final budget in this field.

About the delays produced on tasks, we calculate more time for each task than the strictly necessary in order to prevent delays, but it still can modify the final budget if a big delay is produced. In order to prevent this, we must review how the planning is going after finishing sprint 1, 3 and 5 on the main development task, so any delay can be corrected soon.

Also, we have a little part of the final budget to be used on this type of costs (10\% of contingency), so even if we need more time or a new software, maybe we will not need to modify the final budget anyway.

Lastly if we need to determine deviations, we must use the next equations:
\begin{itemize}
\item Deviation in the developing of a task in costs = (estimated costs - actual cost) * final hours
\item Deviation of a resource in costs = (estimated costs - real cost) * actual consumption
\item Deviation in the developing of a subtask in time = (estimated time - actual hours consumption) * estimated cost
\item Deviation of a resource in time = (estimated consumption - actual consumption) * actual cost
\end{itemize}