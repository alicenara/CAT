\section{Abstract}
Analysing censorship has always been a hard duty, and as there are lots of countries that apply censorship on Internet, a tool is needed to avoid it, or at least to know how.

Tor, a network that let users to be anonymous, sometimes gets blocked by countries that want to control everything their citizen do on the Internet. And each country censors in a different way, with different methods, so is not easy to know how it is being done. 
To solve this problem, PhD Philipp Winter wrote a paper with the specification of a software to analyse censorship on Tor inside the censored country, but up to now, it still has not been developed.

In this document, we will discuss this project scope, its state of the art, its scheduling, the resources needed, the budget requirement and finally we will talk about sustainability.

This project is about programming and avoiding censorship but also it wants to help people to use their rights, like freedom of expression.