\section{Introduction and contextualization}
\label{sec:introduction}

Internet is a huge network used for different purposes: to talk to friends, to get or upload media like videos or pictures, to learn, to teach, to report on what is happening in your city or to do all of this things and more. 

But what happens when a country do not want their people to tell the rest of the world what is going on inside? Or they just want to know what their citizen do on Internet: what sites they visit, who they talk to, which files they upload and what those files are...?

Tor, explained in section \ref{sssec:tor}, offers a way to beat this control efficiently but, as it is an Open Source project, it can be easily blocked by censors. And censorship is not as simple to pass due to the lack of information about how is it done (this systems are black boxes). This situation generates an asymmetry of information that is needed to be reduced.

In 2013, Karlstad University PhD Philipp Winter published a paper called "\textit{Towards a Censorship Analyser for Tor}” (from now on called “\textit{Censorship on Tor paper}”), describing the problems of censorship on Tor network \cite{TorPaper}. It is indeed a trouble because freedom (of speech, of information, etc.) must be guaranteed to everyone, even in the most restrictive countries.

\pagebreak
However, to analyse how the blackout is done outside the country is a very difficult business because most countries block internal traffic but not the external one. The best way to do this is having a machine inside the region (a Virtual Private Network (VPN), volunteer's machines,...) and being really careful not to draw the attention of censors.

The main goal of this project is to build a really useful software (prepared for all level users) that, after running some scripts, will get a lot of information about how countries are blocking Tor in order to find a way to avoid it.

\subimport{introandcontext/}{context.tex}

\pagebreak
\subimport{introandcontext/}{stakeholders.tex}