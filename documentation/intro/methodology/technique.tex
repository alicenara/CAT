\subsection{Custom technique}
To accomplish all goals in this project there is a need of organizing, as there are a lot of work to do, and moreover it is important to use a methodology that allows to quickly detect errors and make changes in the user interface easily as this project needs to be a very robust and easy to use software.

For all that reasons, we will use for the \textit{Main Development} task an iterative and recursive methodology, which will consist in some sprints (iterations) that last 2 weeks approximately. Each sprint will be composed of a developing part and then a testing phase, where we will allow volunteer people (we will ask the Tor community for help but we will accept anyone else too) to test it in order to get feedback not only to fix bugs but to improve the user interface and usability. 

Some of the reasons in choosing this methodology are: to be almost totally sure each feature works before starting the next one, to test security system on each step, to correct code and design errors on time (before having a complete solution, because further modifications are worse) and to have functional prototypes that are prepared to be used even before finishing this project. 

More information about this methodology can be found in the subsection \ref{ssec:tasks} Task description.

All other tasks of this project will be done with a cascade methodology as almost all are information gathering.
