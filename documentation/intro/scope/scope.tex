\subsection{Scope}
\label{ssec:scope}
In order to be able to solve this problem, and basing the solution in the \textit{"Censorship on Tor paper"}, the final software will fulfil the following requirements:
\begin{enumerate}
\item A lot of censoring tests will be provided in the final software. They will be done assuring user's security and respecting its privacy, and users will be allowed to enable (or disable) each one of them. All this test are well described in the \textit{"Censorship on Tor paper"}.
\item The program will be designed to run as much quiet as it can so it will not be easily detected if the machine is seized by the censors (as it might be illegal to run this application).
\item It will have an option to analyse machine's traffic as another tool to evaluate the censorship on the country, but the software will not save any personal or sensitive information.
\item It will be available in multiple platform (Windows, Mac, Linux).
\item It will be presented as a harmless application with a high usability and simple design.
\item It will do some reports and send them to Tor Project so they can be evaluated by the community.
\end{enumerate}

Also we expect to meet the 471 hours planning exposed on section \ref{sec:sched} without delays, and to need no more that the given budged of 11.189,20€ explained on section \ref{sec:budget}.

Finally, each part of this document has its risks covered: on section \ref{ssec:obstacles} (contextualization and scope), on section \ref{ssec:alternativesSchedule} (project schedule) and on section \ref{ssec:budgetControl} (budget evaluation). So even if we do not meet previous scheduling and budget requirements, there are alternatives to finish the project successfully anyway.