\subsection{Previous attempts}
The Tor Project has a big community and is widely known around the world. \textit{"A Lightweight Censorship Analyser"} was one of the projects that Tor community encouraged people to develop, since it would be really useful. Any of the current built projects has all this project features. However, few weeks ago Philipp Winter, the Tor paper author and who carried this project supervision, erased it from the volunteers page as he does not have much time to spend on it anymore. 

But as it was for a lot of years one of the eligible projects for Google Summer of Code (a Google program to encourage students in Open Source developing), it was already started by other students as can be seen in the linked email digest \cite{TorDevMail} and in a Github repository \cite{OtherRepo}. 

With all this information, we conclude this proposal was suitable to become a TFG project for a lot of reasons:
\begin{itemize}
\item It is an Open Source project that will help a community.
\item It needs some specific and technical knowledge in order to program all the scripts to do the tests.
\item It requires some months of coding and testing, so it is long enough.
\item As it is related to network analysis and network resources as communication protocols, services as Domain Name System (DNS), etc., this is associated to Information Technologies specialization, that is what we needed.
\end{itemize}

Although some code of previous attempts can be found on Internet, this project will not use them because it is important to have full control over all its features and the best way to do this is to have a clean and fresh start.
