\subsection{Alternatives and action plan}
\label{ssec:alternativesSchedule}
This project needs to be tested minutely on each sprint before assuming each feature works properly, since a security leak could result in a serious problem, and not only for the volunteer running it but for the application survival: a failure makes the people distrust on the program even after the problem is being fixed.

And testing has been always a really slow task as a lot of people are needed. Not anyone is a candidate: they must be willing to report bugs and need to have a lot of free time. So testing time could be enhanced after each sprint and could modify the final scheduling.

To cope with some volunteers testing this project, we will use the Github report feature so they can explain all bugs found on the project and everything they think it is necessary to be changed, and resolving reports will be a part of the testing phase of each sprint. Also we will notify to Tor community some days before a testing phase will begin so they can be aware of it and help us. 

Finally, if we need some time, there is a sprint that is not really mandatory and consumes a few days: Sprint 6 has as goal to port this project to the most used Operative Systems. This is a hard task to do but not really a must (paper stated that a Windows application would be enough) so it can be done exclusively to a single Operative System or just exclude Mac OSX users.

As can be seen in Gantt project (section \ref{ssec:gantt}), there are some days left after the final date (a total of 3 days), so even though there is a small delay of timings because of the testing phases, the project will still be finished before its final delivering on June \nth{21} .
