\subsubsection{T1 - Viability of the project}
\label{sssec:viability}

\textbf{Summary}
\begin{table}[ht]
\centering
  \begin{tabular}{| c | c | c |}
  \hline Hours assigned & Hours per day & Weeks \\ \hline  
   12 & 4 & only 3 days        \\ \hline
  \end{tabular}
  \caption{Viability of the project hours review} \vspace{3pt}
  \label{tab:viability}
\end{table}

\textbf{Explanation}

Before starting a TFG (Degree Final Project), it is necessary to do a viability study to be sure the project fulfil TFG's requirements: it is possible to do with actual technology, its scope is limited, it will help a community (research community, open source community, etc.), and it can be finished in 3 to 4 months as well.

In this case, this project viability was proven using the following arguments:
\begin{itemize}
\item This project was listed in Tor Volunteers projects so it will help Tor's community. Also its scope is defined by a paper, and since some work (scripts) is already done as a part of another tool, it should be possible to finish in a few months.
\item Moreover we have contacted with the author of the paper in which the project is based, and he kindly accepted to help us sometimes with it.
\item In \textit{"Censorship on Tor paper"}, when talking about the technology that this software will need, it recommended to be developed in Python (that is a widely used programming language) without any special hardware or  software. So we conclude it can be done with just standard technology.
\end{itemize}

This duty was done in only 12 hours counting the time spent doing research, meetings with the TFG director, and the emails sent to Philipp Winter (the author of the paper, explained on \ref{sec:introduction}).
So in a few days this viability study was finished with a positive result, and the next task started after some weeks because of holidays.