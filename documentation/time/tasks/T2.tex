\subsubsection{T2 - Project management}
\label{sssec:projectManagement}

\textbf{Summary}
\begin{table}[ht]
\centering
  \begin{tabular}{| c | c | c |}
  \hline Hours assigned & Hours per day & Weeks \\ \hline  
   89 & 3 & 5 weeks and 3 days        \\ \hline
  \end{tabular}
  \caption{Project management hours review} \vspace{3pt}
  \label{tab:projectManagement}
\end{table}

\par{\textbf{Explanation}}\\
In this task, the project documentation structure is defined and some research is done. When finished, we obtained an extensive document which will be a part of the final documentation delivered at the end of the project.

This part takes at least 75 hours of work, but the final amount of time spent is near 89 hours. Due to external issues, we could only work on this part for three and a half hours per day instead of six, so it was finally built in a month and a week. 

This task is mandatory and have the following elements:
\begin{itemize}
\item Scope of the project and contextualization
\item Time scheduling
\item Economic management and sustainability
\item Others (tasks to be done that are not reflected in the final document)
\begin{itemize}
	\item First oral presentation
	\item Technical specifications
	\item Final document
	\item Final presentation
\end{itemize}
\end{itemize}

As we can observe in the Gantt chart on section \ref{ssec:gantt}, this part seems to take longer than five weeks but it is because there is a holiday week (from 21 to 24 of March) and also the final presentation point on others has a week assigned but it will only need 15min per day as it will only be a preparation for the final presentation.

It provides a solid start because it clearly defines how the project will be done, deadlines, milestones, needed resources, etc. So all other tasks (excluding the viability of the project, \ref{sssec:viability}) are defined and scheduled in this one.