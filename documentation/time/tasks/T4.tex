\subsubsection{T4 - Main development}
\label{sssec:mainDev}

\textbf{Summary}
\begin{table}[ht]
\centering
  \begin{tabular}{| c | c | c |}
  \hline Hours assigned & Hours per day & Weeks \\ \hline  
  306  & Variable & 9        \\ \hline
  \end{tabular}
  \caption{Main development hours review} \vspace{3pt}
  \label{tab:mainDev}
\end{table}

\par{\textbf{Explanation}}\\
In this section, some tasks (further specified) will be done using a recursive and incremental methodology, as it allows to discover problems early and help the project adapt to client requirements.

The workflow of this technique is composed by sprints (or iterations), which are the basic unit of development. A sprint is a sequence of procedures with a defined duration (1 to 4 weeks). During this lapse, it is important to first determine the sprint scope and divide all work in small tasks. Then, it is necessary to decide what will be done during each day at the beginning of it.

Finally, when the sprint ends, a review of completed tasks, problems, what could be improved and all testing reports (as testing is also a part of the sprint) will be documented.

To follow this methodology specification, from the third sprint each of the following ones will be divided into 3 different stages: Development, Critical testing, Testing and Documentation. 
\begin{itemize}
\item Development stage will start parallel to testing phase of the preceding sprint (if exists), and will be the longer phase of the sprint.
\item Critical testing and Documentation will start both once that sprint Development phase is already done, will last 2 days and will be done in parallel.
\item And finally Testing stage will start after Critical testing is finished and will last three days. 
\end{itemize}

This project sprints are described below.

\pagebreak
\subimport{./}{T4S1.tex}\\


\subimport{./}{T4S2.tex}\\


\subimport{./}{T4S3.tex}\\

\pagebreak
\subimport{./}{T4S4.tex}\\


\subimport{./}{T4S5.tex}\\


\subimport{./}{T4S6.tex}